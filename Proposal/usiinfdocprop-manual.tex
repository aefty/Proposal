\documentclass[a4paper]{article}
\usepackage{booktabs}
\usepackage{url,xspace,array}
\newcommand{\clsusiinf}{\textsf{usiinfdocprop}\xspace}
\author{ Domenico Bianculli}
\title{\clsusiinf Doctoral Proposal Style
  Documentation\footnote{\textsf{usiinf} and USI-INF
  stand for \emph{Universit\`a della Svizzera Italiana (USI)  - Faculty of Informatics}.}}
\setlength{\marginparwidth}{1in}
\begin{document}
\maketitle

\section{Overview}

The \clsusiinf\LaTeX\ class for USI-INF doctoral proposals is based on the standard
\textsf{book} class, and provides additional commands, environments
and default text to help typesetting  proposals submitted at USI-INF.

\section{Minimal Document Structure}

The minimal structure of a dissertation document is shown below:
\begin{verbatim}

\documentclass[]{usiinfdocprop}

\title{My Doctoral Proposal} 
\author{Philo S. Doctor}
\begin{committee}
  \advisor{Prof.}{Alonzo}{Church}{University of California, Los
    Angeles, USA}
  \internalmember{Dr.}{Leslie}{Lamport}
  % other internal members
  \externalmember{Prof.}{Alan M.}{Turing}{Princeton University, USA}
  % other external members
} 
\phddirector{The PhD program Director \emph{pro tempore}} 

\abstract{Your Abstract}

\begin{document}
\maketitle 

\frontmatter 

\tableofcontents 

\mainmatter

\chapter{Introduction}

\backmatter

\bibliographystyle{plainnat}
\bibliography{biblio}

\end{document}
\end{verbatim}

The commands in the preamble should be self-explanatory and are
required to complete the title page and the list of your 
committee members. See next section for a complete command reference.


\section{Command reference}

\reversemarginpar

This command reference lists only commands provided by the USI-INF
doctoral proposal class. Commands inherited from the \texttt{book} class or
any of the loaded packages are not documented here, unless 
significant changes were made to them.

\subsection{Preamble Commands and Environments}
\noindent
The title of the proposal\marginpar{\texttt{\char`\\ title}}. This
command is mandatory and takes one argument.

\medskip

\noindent
The subtitle of the proposal\marginpar{\texttt{\char`\\subtitle}}. This
command is optional and takes one argument.

\medskip

\noindent
 The author of this proposal, i.e.,
you. This\marginpar{\texttt{\char`\\ author}} 
command is mandatory and takes one argument.

\medskip

\noindent
This environment constructs the list of your committee members.
\marginpar{\texttt{committee}} 
Inside this environment you may use the \verb!\advisor!,
\verb!\coadvisor!, \verb!internalmember!, \verb!\externalmember! commands.
This environment is mandatory.

\medskip

\noindent
The research advisor, or main advisor\marginpar{\texttt{\char`\\ advisor}}
responsible for the student submitting the proposal. This command is
mandatory and takes four arguments:  academic
title, first and middle name, last name and affiliation. 

\medskip

\noindent
 The co-advisor of the student. This\marginpar{\texttt{\char`\\coadvisor}} 
command is optional and should only be used if you have an official
co-advisor. This command  takes four arguments:  academic
title, first and middle name, last name and affiliation. 

\medskip

\noindent
Creates an entry for one internal 
\marginpar{\texttt{\char`\\internalmember}}
member in the list of your
committee. This commands takes three arguments:  academic title,
 first and middle name, last name; an optional argument can be given
to specify an affiliation different than USI-INF (e.g., an affiliated
research institute).

\medskip

\noindent
Creates an entry for one external 
\marginpar{\texttt{\char`\\externalmember}}
member in the list of your
committee. This commands takes four arguments:  academic title,
 first and middle name, last name, affiliation.


\medskip

\noindent
The director of the USI-INF PhD
program.\marginpar{\texttt{\char`\\phddirector}} This is mandatory
and takes one argument.

\medskip

\noindent
Typeset the abstract on the title page.
\marginpar{\texttt{\char`\\abstract}} This command is mandatory and
takes one argument.

\medskip

\subsection{Text Body Commands and Environments}
\noindent
Typeset the committee on a dedicated page \marginpar{\texttt{\char`\\frontmatter}} and initializes formatting and
pagination settings. This command \emph{must} appear after
\verb|\maketitle| and before any other commands or text.

\medskip

\noindent
This command must appear after
all\marginpar{\texttt{\char`\\mainmatter}} frontmatter parts (e.g.,
list of figures, list of tables) and before the
first chapter of the main text body.

\medskip

\noindent
This command switches formatting\marginpar{\texttt{\char`\\backmatter}} and pagination to the form used for the
backmatter. This command \emph{must} appear after the last
chapter/appendix and before the bibliographic references.

\subsection{Class Options}

The default layout produced by the class is targeted to ``electronic''
\marginpar{\texttt{print}} publishing and uses margins consistent with the normal \LaTeX\
\texttt{oneside} option. This option
 switches the layout and various other things to something that is
more suitable for two-sided printing and binding.
 Standard \LaTeX\ options \texttt{oneside} or \texttt{twoside} are disabled.

\medskip

\noindent
By default, the class loads the \texttt{hyperref} package with the proper
options. \marginpar{\texttt{nohyper}} Since the \texttt{hyperref} package redefines
many \LaTeX\ commands, it may conflict with other packages you
use. This option let you  disable the loading of the package.



\section{Required Packages}

The \clsusiinf class makes extensive use of a wide range of
``standard''\footnote{Standard packages are the ones available in a
  modern \LaTeX\ distribution, like \TeX Live
  (\url{http://www.tug.org/texlive}) and Mac\TeX\ (\url{http://www.tug.org/mactex}).}
\LaTeX\ packages. Table~\ref{tab:packages} lists all packages (and options)
that are loaded by the class, and thus do not need to be loaded in
your thesis document.

\begin{table}
\centering
\begin{tabular}{>{\sffamily}l>{\ttfamily}l}
\toprule
Package & Options\\
\midrule
amsmath & \\
book (class) & a4paper, 12pt, onecolumn, final, openright, titlepage
\\
beramono & scaled \\
booktabs & \\
calc & \\
caption & font=sf, labelsep=period \\
datatool & \\
fancyhdr & \\
fontenc & T1\\
geometry & a4paper \\
graphicx & \\
hyperref &  unicode, plainpages=false, pdfpagelabels, breaklinks\\
hypcap & all \\
mathdesign & charter\\
natbib & square \\
sectsty & \\
textcomp & \\
url & \\

\bottomrule
\end{tabular}
\caption{Required packages and selected options}\label{tab:packages}
\end{table}

The packages \texttt{beramono} and \texttt{mathdesign} select, respectively, the
monospaced and the math fonts.
The class file also uses the \emph{Optima} font package for the sans
serif fonts\footnote{The \emph{Optima} (aka \emph{URW Classico}) font
  is not bundled with \TeX live-based distributions. However, they
  provide a script, \texttt{getnonfreefonts}, for installing extra
  fonts. To install \emph{Optima}, just type \texttt{getnonfreefonts
    classico} on the command line; the script requires \texttt{wget}
  to be installed.}. \texttt{fontenc} and \texttt{textcomp} are required to
make these fonts work properly. 


\section{Acknowledgments}
\clsusiinf is based on the \textsf{usiinfthesis} class, written by
Domenico Bianculli and Jochen Wuttke.

\section{Version History}
\begin{description}
\item[2009/08/25 v.~1.0.2] \ \\  Fixed compatibility with
  \textsf{datatool} v.~2.0.x
\item[2009/07/05 v.~1.0.1] \ \\  Included changes of
  \textsf{usiinfthesis} v.~1.0.7: Fixed header capitalization problems
  in frontmatter. Fixed page number issue with toc, lof, lot
\item[2009/06/23 v.~1.0] initial release
\end{description}

\end{document}

